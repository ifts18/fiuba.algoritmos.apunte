% Copyright (C) Rosita Wachenchauzer <rositaw@gmail.com>
% Copyright (C) Margarita Manterola <margamanterola@gmail.com>

% Esta obra está licenciada de forma dual, bajo las licencias Creative
% Commons:
%  * Atribución-Compartir Obras Derivadas Igual 2.5 Argentina
%    http://creativecommons.org/licenses/by-sa/2.5/ar/
%  * Atribución-Compartir Obras Derivadas Igual 3.0 Unported
%    http://creativecommons.org/licenses/by-sa/3.0/deed.es_AR.
%
% A su criterio, puede utilizar una u otra licencia, o las dos.
% Para ver una copia de las licencias, puede visitar los sitios
% mencionados, o enviar una carta a Creative Commons,
% 171 Second Street, Suite 300, San Francisco, California, 94105, USA.

\chapter[Funciones]{Funciones}

En la primera unidad vimos que el programador puede definir nuevas
instrucciones, que llamamos funciones. En particular lo aplicamos a la
construcción de una función llamada \lstinline+hola+ que salude a todos a
quienes queramos saludar:

\begin{codigo-python-sn}
def hola(alguien):
    print "Hola ", alguien, "!"
    print "Estoy programando en Python."
\end{codigo-python-sn}

Dijimos en esa ocasión que las funciones tienen partes variables, llamadas
parámetros, que se asocian a un valor distinto en cada invocación.  El valor
con el que se asocia un parámetro se llama {\it argumento}.  En nuestro caso la
invocamos dos veces, para saludar a Ana y a Juan, haciendo que
\lstinline+alguien+ se asocie al valor \lstinline!"Ana"! en la primera llamada
y al valor \lstinline!"Juan"! en la segunda:

\begin{codigo-python-sn}
>>> hola("Ana")
Hola Ana !
Estoy programando en Python.
>>> hola("Juan")
Hola Juan !
Estoy programando en Python.
>>>
\end{codigo-python-sn}

Una función puede tener ninguno, uno o más parámetros. La función
\lstinline+hola+ tiene un parámetro. Ya vimos también ejemplos de funciones sin
parámetros:

\begin{codigo-python-sn}
def holaPab():
    print "Hola Pablo!"
    print "Estoy programando en Python."
\end{codigo-python-sn}

En el caso de tener más de un parámetro, éstos se separan entre sí por comas,
y en la invocación también se separan por comas los argumentos.

\section{Documentación de funciones}

Cada función escrita por un programador realiza una tarea específica.  Cuando
la cantidad de funciones disponibles para ser utilizadas es grande, puede ser
difícil saber exactamente qué hace una función.  Es por eso que es
extremadamente importante documentar en cada función cuál es la tarea que
realiza, cuáles son los parámetros que recibe y qué es lo que devuelve, para
que a la hora de utilizarla sea lo pueda hacer correctamente.

La documentación de una función se coloca luego del encabezado de la función,
en un párrafo encerrado entre \lstinline!"""!.  Así, para la función vista en
el ejemplo anterior:

\begin{codigo-python-sn}
def hola(alguien):
    """ Imprime por pantalla un saludo, dirigido a la persona que
        se indica por parámetro. """
    print "Hola ", alguien, "!"
    print "Estoy programando en Python."
\end{codigo-python-sn}

Cuando una función definida está correctamente documentada, es posible acceder
a su documentación mediante la función \lstinline!help! provista por Python:

\begin{codigo-python-sn}
>>> help(hola)
Help on function hola in module __main__:

hola(alguien)
    Imprime por pantalla un saludo, dirigido a la persona que
    se indica por parámetro.
\end{codigo-python-sn}

De esta forma no es necesario mirar el código de una función para saber lo que
hace, simplemente llamando a \lstinline!help! es posible obtener esta
información.

\section{Imprimir versus Devolver}

A continuación se define una función
\lstinline+print_asegundos (horas, minutos, segundos)+
con tres parámetros (\lstinline+horas+, \lstinline+minutos+
y \lstinline+segundos+) que imprime por pantalla la transformación a segundos
de una medida de tiempo expresada en horas, minutos y segundos:

\begin{codigo-python}
def print_asegundos (horas, minutos, segundos):
    """ Transforma en segundos una medida de tiempo expresada en
        horas, minutos y segundos """
    segsal = 3600 * horas + 60 * minutos + segundos  # regla de transformación
    print "Son", segsal, "segundos"
\end{codigo-python}

Para ver si realmente funciona, podemos ejecutar la función de la siguiente
forma:

\begin{codigo-python-sn}
>>> print_asegundos (1, 10, 10)
Son 4210 segundos
\end{codigo-python-sn}

Contar con funciones es de gran utilidad, ya que nos permite ir armando una
biblioteca de instrucciones con problemas que vamos resolviendo, y que se
pueden reutilizar en la resolución de nuevos problemas (como partes de un
problema más grande, por ejemplo) tal como lo sugiere Thompson en ``How to
program it''.

Sin embargo, más útil que tener una biblioteca donde los resultados
se imprimen por pantalla, es contar con una biblioteca donde los
resultados se devuelven, para que la gente que usa esas funciones manipule
esos resultados a voluntad: los imprima, los use para realizar cálculos
más complejos, etc.

\begin{codigo-python}
def calc_asegundos (horas, minutos, segundos):
    """ Transforma en segundos una medida de tiempo expresada en
        horas, minutos y segundos """
    segsal = 3600 * horas + 60 * minutos + segundos  # regla de transformacion
    return segsal
\end{codigo-python}

De esta forma, es posible realizar distintas operaciones con el valor obtenido
luego de hacer la cuenta:

\begin{codigo-python-sn}
>>> print calc_asegundos (1, 10, 10)
4210
>>> print "Son", calc_asegundos (1, 10, 10), "segundos"
Son 4210 segundos
>>> y = calc_asegundos(1, 10, 10)
>>> z = calc_asegundos(2, 20, 20)
>>> y+z
12630
\end{codigo-python-sn}

\ejercicioc{Escribir una función \lstinline+repite_hola+ que reciba como
parámetro un número entero \lstinline+n+ y escriba por pantalla el mensaje
\lstinline!"Hola"!  \lstinline+n+ veces.  Invocarla con distintos valores de
\lstinline+n+.}

\ejercicioc{Escribir otra función \lstinline+repite_hola+ que reciba como
parámetro un número entero \lstinline+n+ y retorne la cadena formada por
\lstinline+n+ concatenaciones de  \lstinline!"Hola"!. Invocarla con distintos
valores de \lstinline+n+.}

\ejercicioc{Escribir una función \lstinline+repite_saludo+ que reciba como
parámetro un número entero \lstinline+n+ y una cadena \lstinline+saludo+ y
escriba por pantalla el valor de \lstinline+saludo+ \lstinline+n+ veces.
Invocarla con distintos valores de \lstinline+n+ y de \lstinline+saludo+.}

\ejercicioc{Escribir otra función \lstinline+repite_saludo+ que reciba como
parámetro un número entero \lstinline+n+ y una cadena \lstinline+saludo+
retorne el valor de \lstinline+n+ concatenaciones de \lstinline+saludo+.
Invocarla con distintos valores de \lstinline+n+ y de \lstinline+saludo+.}

\section{Cómo usar una función en un programa}

Una función es útil porque nos permite repetir la misma
instrucción (puede que con argumentos distintos) todas las veces
que las necesitemos en un programa.

Para utilizar las funciones definidas anteriormente, escribiremos un programa
que pida tres duraciones, y en los tres casos las transforme a segundos y las
muestra por pantalla.

\begin{enumerate}

\item {\bf Análisis: } El programa debe pedir tres duraciones expresadas en
horas, minutos y segundos, y las tiene que mostrar en pantalla expresadas en
segundos.

\item {\bf Especificación: }
\begin{itemize}
\item {\bf Entradas: } Tres duraciones leídas de teclado y expresadas en horas,
minutos y segundos.
\item {\bf Salidas: } Mostrar por pantalla cada una de las duraciones
ingresadas, convertidas a segundos.  Para cada juego de datos de entrada (h, m,
s) se obtiene entonces 3600 * h + 60 * m + s, y se muestra ese resultado por
pantalla.
\end{itemize}
\item {\bf Diseño:}
\begin{itemize}
\item Se tienen que leer tres conjuntos de datos y para cada conjunto hacer lo
mismo, se trata entonces de un programa con estructura de ciclo definido de
tres pasos:

\begin{verbatim}
repetir 3 veces:
    <hacer cosas>
\end{verbatim}

\item El cuerpo del ciclo ( \verb+<hacer cosas>+)  tiene la estructura {\it
Entrada-Cálculo-Salida}.  En pseudocódigo:

\begin{verbatim}
Leer cuántas horas tiene el tiempo dado
 (y referenciarlo con la variable hs)

Leer cuántos minutos tiene tiene el tiempo dado
 (y referenciarlo con la variable min)

Leer cuántos segundos tiene el tiempo dado
 (y referenciarlo con la variable seg)

Mostrar por pantalla 3600 * hs + 60 * min + seg
\end{verbatim}

Pero convertir y mostrar por pantalla es exactamente lo que
hace nuestra función \verb+print_asegundos+, por lo que podemos
hacer que el cuerpo del ciclo se diseñe como:

\begin{verbatim}
Leer cuántas horas tiene la duración dada
 (y referenciarlo con la variable hs)

Leer cuántos minutos tiene tiene la duración dada
 (y referenciarlo con la variable min)

Leer cuántas segundos tiene la duración dada
 (y referenciarlo con la variable seg)

Invocar la función print_asegundos(hs, min, seg)
\end{verbatim}

\item El pseudocódigo final queda:

\begin{verbatim}
repetir 3 veces:
        Leer cuántas horas tiene la duración dada
        (y referenciarlo con la variable hs)

        Leer cuántos minutos tiene la duración dada
        (y referenciarlo con la variable min)

        Leer cuántos segundos tiene la duración dada
        (y referenciarlo con la variable seg)

        Invocar la función print_asegundos(hs, min, seg)
\end{verbatim}

\end{itemize}
\item {\bf Implementación:} A partir del diseño, se escribe el programa
Python que se muestra en el Código \ref{trestiempos}, que se guardará
en el archivo \verb!tres_tiempos.py!.

\begin{codigo}{\label{trestiempos} tres\_tiempos.py}{Lee tres tiempos y los imprime en segundos}
\begin{codigo-python}
def print_asegundos (horas, minutos, segundos):
    """ Transforma en segundos una medida de tiempo expresada en
        horas, minutos y segundos """
    segsal = 3600 * horas + 60 * minutos + segundos
    print "Son", segsal, "segundos"

def main():
	""" Lee tres tiempos expresados en hs, min y seg, y usa
		print_asegundos para mostrar en pantalla la conversión a
        segundos """
    for x in range(3):
        hs = input("Cuantas horas?: ")
        min = input("Cuantos minutos?: ")
        seg = input("Cuantos segundos?: ")
        print_asegundos(hs, min, seg)

main()
\end{codigo-python}
\end{codigo}

\item {\bf Prueba: } Probamos el programa con las ternas (1,0,0), (0,1,0) y
(0,0,1):

\begin{codigo-python-sn}
>>> import tres_tiempos
Cuantas horas?: 1
Cuantos minutos?: 0
Cuantos segundos?: 0
Son 3600 segundos
Cuantas horas?: 0
Cuantos minutos?: 1
Cuantos segundos?: 0
Son 60 segundos
Cuantas horas?: 0
Cuantos minutos?: 0
Cuantos segundos?: 1
Son 1 segundos
>>>
\end{codigo-python-sn}
\end{enumerate}

\ejercicioc{Resolver el problema anterior usando ahora la función
\verb+calc_asegundos+.}

\section{Más sobre los resultados de las funciones}

Ya hemos visto cómo hacer para que las funciones que se comporten como las
funciones que conocemos, las de la matemática, que
se usan para calcular resultados.

Veremos ahora varias cuestiones a tener en cuenta al escribir
funciones. Para ello volvemos a escribir una función que eleva al cuadrado un número.

\begin{codigo-python-sn}
>>> def cuadrado (x):
...         cua = x * x
...         return cua
...
>>> y = cuadrado (5)
>>> y
25
>>>
\end{codigo-python-sn}

¿Por qué no usamos dentro del programa el valor \verb+cua+ calculado dentro de
la función?

\begin{codigo-python-sn}
>>> def cuadrado (x):
...     cua = x * x
...     return cua
...
>>> cua
Traceback (most recent call last):
  File "<stdin>", line 1, in <module>
NameError: name 'cua' is not defined
>>>
\end{codigo-python-sn}

\begin{observacion}
Las variables y los parámetros que se declaran
dentro de una función no existen fuera de ella, no se los conoce.
Fuera de la función se puede ver sólo el valor que retorna
y es por eso que es necesario introducir la instrucción \lstinline!return!.
\end{observacion}

¿Para qué hay que introducir un \lstinline+return+ en la función?
¿No alcanza con el valor que se calcula dentro de la misma
para que se considere que la función retorna un valor? En
Python no alcanza (hay otros lenguajes en los que se considera que el
último valor calculado en una función es el valor de retorno de la misma).

\begin{codigo-python-sn}
>>> def cuadrado (x):
...     cua = x * x
...
>>> y = cuadrado (5)
>>> y
>>>
\end{codigo-python-sn}

Cuando se invoca la función \lstinline!cuadrado! mediante la instrucción
\verb+y = cuadrado (5)+ lo que sucede es lo siguiente:

\begin{itemize}
\item Se invoca a \lstinline!cuadrado! con el argumento \lstinline!5!, y se ejecuta
el cuerpo de la función.
\item El valor que devuelve la función se asocia con la variable \lstinline!y!.
\end{itemize}

Es por eso que si la función no devuelve ningún valor, no queda ningún valor
asociado a la variable \lstinline!y!.

\section{Un ejemplo completo}

\begin{problemac}
Un usuario nos plantea su problema: necesita que se facture el uso de un teléfono.
Nos informará la tarifa por segundo, cuántas comunicaciones se realizaron,
la duración de cada comunicación expresada en horas, minutos y segundos.
Como resultado deberemos informar la duración en segundos de cada comunicación y
su costo.
\end{problemac}

\begin{solucion}
Aplicaremos los pasos aprendidos:

\begin{enumerate}

\item {\bf Análisis: }
\begin{itemize}
\item ¿Cuántas tarifas distintas se usan? Una sola (la llamaremos \lstinline!f!).
\item ¿Cuántas comunicaciones se realizaron? La cantidad de comunicaciones (a
la que llamaremos \lstinline!n!) se informa cuando se inicia el programa.
\item ¿En qué formato vienen las duraciones de las comunicaciones? Vienen como ternas (h, m, s).
\item ¿Qué se hace con esas ternas? Se convierten a segundos y se calcula el costo de cada
comunicación multiplicando el tiempo por la tarifa.
\end{itemize}

\item {\bf Especificación: }
\begin{itemize}

\item {\bf Entradas: }
\begin{itemize}
\item Una tarifa \lstinline+f+ expresada en pesos/segundo.
\item Una cantidad \lstinline+n+ de llamadas telefónicas.
\item \lstinline+n+ duraciones de llamadas leídas de teclado y expresadas en horas, minutos y segundos.
\end{itemize}

\item {\bf Salidas: } Mostrar por pantalla las \lstinline+n+ duraciones ingresadas, convertidas a segundos,
y su costo.
Para cada juego de datos de entrada (h, m, s) se imprime:
\begin{displaymath}
3600 * h + 60 * m + s, f * (3600 * h + 60 * m + s) .
\end{displaymath}
\end{itemize}

\item {\bf Diseño:}

Siguiendo a Thompson, buscamos un programa que haga algo análogo, y vemos si se
lo puede modificar para resolver nuestro problema.  El programa
\lstinline+tres_tiempos+ que hicimos anteriormente,  se parece bastante a lo
que necesitamos. Veamos las diferencias entre sus especificaciones.

\begin{tabular}[l]{|p{6.8cm}|p{7.5cm}|}
\hline {\lstinline!tres_tiempos.py!} & \lstinline!tarifador.py!\\
\hline
\begin{verbatim}
repetir 3 veces:
    <hacer cosas>
\end{verbatim}
&
\begin{verbatim}
leer el valor de f
leer el valor de n
repetir n veces:
    <hacer cosas>
\end{verbatim}
\\
\hline
El cuerpo del ciclo:
{\footnotesize
\begin{verbatim}
Leer el valor de hs
Leer el valor de min
Leer el valor de seg
Invocar print_asegundos(hs, min, seg)


\end{verbatim}
} &
El cuerpo del ciclo:
{\footnotesize
\begin{verbatim}
Leer el valor de hs
Leer el valor de min
Leer el valor de seg
Asignar segcalc = asegundos(hs, min, seg)
Calcular costo = segcalc * f
Mostrar por pantalla segcalc y costo
\end{verbatim}
} \\
\hline
{\footnotesize
\begin{verbatim}
print_asegundos (hs, min, seg):
	segsal = 3600*hs+60*min+seg
	print segsal
\end{verbatim}
} &
{\footnotesize
\begin{verbatim}
asegundos (hs, min, seg):
	segsal = 3600*hs+60*min+seg
	return segsal
\end{verbatim}
} \\
\hline
\end{tabular}


En primer lugar se observa que el \lstinline!tarifador! debe leer el valor de
la tarifa (\lstinline!f!) y que en \lstinline!tres_tiempos! se conoce la
cantidad de ternas (3), mientras que en \lstinline!tarifador! la cantidad
de ternas es un dato a ingresar.

Además, se puede ver que en el cuerpo del ciclo de
\lstinline!tres_tiempos!, se lee una terna y se llama a
\lstinline!print_asegundos! que calcula, imprime y no devuelve ningún
valor.  Si hiciéramos lo mismo en \lstinline!tarifador!, no podríamos
calcular el costo de la comunicación.  Es por ello que en lugar de usar
\lstinline!print_asegundos! se utiliza la función \lstinline!asegundos!,
que calcula el valor transformado y lo devuelve en lugar de imprimirlo y en
el cuerpo principal del programa se imprime el tiempo junto con el costo
asociado.

\item {\bf Implementación:} El siguiente es el programa resultante:

\begin{codigo}{tarifador.py}{Factura el tiempo de uso de un teléfono}
\begin{codigo-python}
def main():
	""" El usuario ingresa la tarifa por segundo, cuántas
		comunicaciones se realizaron, y la duracion de cada
		comunicación expresada en horas, minutos y segundos. Como
		resultado se informa la duración en segundos de cada
        comunicación y su costo. """

    f = input("¿Cuánto cuesta 1 segundo de comunicacion?: ")
    n = input("¿Cuántas comunicaciones hubo?: ")
    for x in range(n):
        hs = input("¿Cuántas horas?: ")
        min = input("¿Cuántos minutos?: ")
        seg = input("¿Cuántos segundos?: ")
        segcalc = asegundos(hs, min, seg)
        costo = segcalc * f
        print "Duracion: ", segcalc, "segundos. Costo: ",costo, "$."

def asegundos (horas, minutos, segundos):
           segsal = 3600 * horas + 60 * minutos + segundos
           return segsal

main()
\end{codigo-python}
\end{codigo}

\item {\bf Prueba:} Lo probamos con una tarifa de \$ 0,40 el segundo y tres
ternas de \lstinline!(1,0,0)!, \lstinline!(0,1,0)! y \lstinline!(0,0,1)!. Ésta
es la corrida:
\begin{codigo-python-sn}
>>> import tarifador
Cuanto cuesta 1 segundo de comunicacion?: 0.40
Cuantas comunicaciones hubo?: 3
Cuantas horas?: 1
Cuantos minutos?: 0
Cuantos segundos?: 0
Duracion:  3600 segundos. Costo:  1440.0 $.
Cuantas horas?: 0
Cuantos minutos?: 1
Cuantos segundos?: 0
Duracion:  60 segundos. Costo:  24.0 $.
Cuantas horas?: 0
Cuantos minutos?: 0
Cuantos segundos?: 1
Duracion:  1 segundos. Costo:  0.4 $.
>>>
\end{codigo-python-sn}

\item {\bf Mantenimiento:}

\ejercicioc {Corregir el programa para que:
\begin{itemize}
\item Imprima el costo en pesos y centavos, en lugar de un número decimal.
\item Informe además cuál fue el total facturado en la corrida.
\end{itemize}
}
\end{enumerate}
\end{solucion}

\section{Devolver múltiples resultados}
\label{fun:multiple_return}
Ahora nos piden que escribamos una función que dada una duración
en segundos sin fracciones (representada por un número entero) calcule
la misma duración en horas, minutos y segundos.

La especificación es sencilla:
\begin{itemize}
\item La cantidad de horas es la duración informada en segundos dividida
por 3600 (división entera).
\item La cantidad de minutos es el resto de la división del paso 1,
dividido por 60 (división entera).
\item La cantidad de segundos es el resto de la división del paso 2.
\item Es importante notar que si la duración no se informa como un número
entero, todas las operaciones que se indican más arriba carecen de sentido.
\end{itemize}

¿Cómo hacemos para devolver más de un valor? En realidad lo que se espera
de esta función es que devuelva una terna de valores: si ya calculamos
\lstinline!hs!, \lstinline!min! y \lstinline!seg!, lo que debemos retornar
es la terna \lstinline+(hs, min, seg)+:

\begin{codigo-python}
def aHsMinSeg (x):
   """ Dada una duración en segundos sin fracciones
      (la función debe invocarse con números enteros)
      se la convierte a horas, minutos y segundos """
   hs = x / 3600
   min = (x % 3600) / 60
   seg = (x % 3600 ) % 60
   return (hs, min, seg)
\end{codigo-python}

Esto es lo que sucede al invocar esta función:

\begin{codigo-python-sn}
>>> (h, m, s) = aHsMinSeg(3661)
>>> print "Son",h,"horas",m,"minutos",s,"segundos"
Son 1 horas 1 minutos 1 segundos
>>> (h, m, s) = aHsMinSeg(3661.0)  # aca violamos la especificacion
>>> print "Son",h,"horas",m,"minutos",s,"segundos" # y esto es lo que pasa:
Son 1.0169444444444444 horas 1.0166666666666666 minutos 1.0 segundos
>>>
\end{codigo-python-sn}

\begin{sabias_que}
Cuando la función debe retornar múltiples resultados se empaquetan todos juntos
en una n-upla del tamaño adecuado.

Esta característica está presente en Python, Haskell, y algunos otros pocos
lenguajes.  En los lenguajes en los que esta característica no está
presente, como C, Pascal, Java o PHP, es necesario recurrir a otras
técnicas más complejas para poder obtener un comportamiento similar.
\end{sabias_que}

Respecto de la variable que hará referencia al resultado de la invocación,
se podrá usar tanto una n-upla de variables como en el ejemplo anterior,
en cuyo caso podremos nombrar en forma separada cada uno de los resultados,
o bien se podrá usar una sola variable, en cuyo caso se considerará que
el resultado tiene un solo nombre y la forma de una n-upla:

\begin{codigo-python-sn}
>>> t=aHsMinSeg(3661)
>>> print t
(1, 1, 1)
>>>
\end{codigo-python-sn}

\begin{observacion}
Si se usa una n-upla de variables para referirse a un resultado,
la cantidad de variables tiene que coincidir con la cantidad de valores que
se retornan.

\begin{codigo-python-sn}
>>> (x,y)=aHsMinSeg(3661)
Traceback (most recent call last):
  File "<stdin>", line 1, in <module>
ValueError: too many values to unpack
>>> (x,y,w,z)=aHsMinSeg(3661)
Traceback (most recent call last):
  File "<stdin>", line 1, in <module>
ValueError: need more than 3 values to unpack
>>>
\end{codigo-python-sn}
\end{observacion}

\section{Resumen}

\begin{itemize}
\item Una función puede tener ninguno, uno o más parámetros.  En el
caso de tener más de uno, se separan por comas tanto en la declaración
de la función como en la invocación.
\item Es altamente recomendable documentar cada función que se
escribe, para poder saber qué parámetros recibe, qué devuelve y qué
hace sin necesidad de leer el código.
\item Las funciones pueden imprimir mensajes para comunicarlos al
usuario, y/o devolver valores.  Cuando una función realice un cálculo
o una operación con sus parámetros, es recomendable que devuelva el
resultado en lugar de imprimirlo, permitiendo realizar otras
operaciones con él.
\item No es posible acceder a las variables definidas dentro de una
función desde el programa principal, si se quiere utilizar algún
valor calculado en la función, será necesario devolverlo.
\item Si una función no devuelve nada, por más que se la asigne a una
variable, no quedará ningún valor asociado a esa variable.
\end{itemize}

\begin{referencia_python}

\begin{sintaxis}{\lstinline!def funcion(param1, param2, param3):!}
Permite definir funciones, que pueden tener ninguno, uno o más
parámetros.  El cuerpo de la función debe estar un nivel de indentación
más adentro que la declaración de la función.

\begin{codigo-python-sn}
def funcion(param1, param2, param3):
    # hacer algo con los parametros
\end{codigo-python-sn}
\end{sintaxis}

\begin{sintaxis}{Documentación de funciones}
Si en la primera línea de la función se ingresa un comentario
encerrado entre comillas, este comentario pasa a ser la documentación
de la función, que puede ser accedida mendiante el comando
\lstinline!help(funcion)!.
\begin{codigo-python-sn}
def funcion():
	""" Esta es la documentación de la función """
	# hacer algo
\end{codigo-python-sn}
\end{sintaxis}

\begin{sintaxis}{\lstinline!return valor!}
Dentro de una función se utiliza la instrucción \lstinline!return!
para indicar el valor que la función debe devolver. \\

Una vez que se ejecuta esta instrucción, se termina la ejecución de la
función, sin importar si es la última línea o no. \\

Si la función no contiene esta instrucción, no devuelve nada.
\end{sintaxis}

\begin{sintaxis}{\lstinline!return (valor1, valor2, valor3)!}
Si se desea devolver más de un valor, se los {\textit empaqueta} en
una tupla de valores.  Esta tupla puede o no ser desempaquetada al
invocar la función:
\begin{codigo-python-sn}
def f(valor):
	# operar
	return (a1, a2, a3)

# desempaquetado:
v1, v2, v3 = f(x)
# empaquetado
v = f(y)
\end{codigo-python-sn}
\end{sintaxis}

\end{referencia_python}

\newpage
\section{Ejercicios}

\extractionlabel{guia}
\begin{ejercicio} Escribir dos funciones que permitan calcular:
\begin{partes}
    \item La duración en segundos de un intervalo dado en horas, minutos y segundos.
    \item La duración en horas, minutos y segundos de un intervalo dado en segundos.
\end{partes}
\end{ejercicio}

\extractionlabel{guia}
\begin{ejercicio}
Usando las funciones del ejercicio anterior, escribir un programa que pida al
usuario dos intervalos expresados en horas, minutos y segundos, sume sus
duraciones, y muestre por pantalla la duración total en horas, minutos y segundos.
\end{ejercicio}

\extractionlabel{guia}
\begin{ejercicio}
Escribir una función que, dados cuatro números, devuelva el mayor
producto de dos de ellos. Por ejemplo, si recibe los números 1, 5, -2,
-4 debe devolver 8, que es el producto más grande que se puede obtener
entre ellos ($8 = -2 \times -4$).
\end{ejercicio}

\extractionlabel{guia}
\begin{ejercicio}
{\bf Área de un triángulo en base a sus puntos}
\begin{partes}

    \item Escribir una función que dado un vector al origen (definido por sus
 coordenadas \verb!x,y!), devuelva la norma del vector, dada por
 $||\vec{(x,y)}||=\sqrt{x^2+y^2}$

    \item Escribir una función que dados dos puntos en el plano (\verb!x1,y1! y
 \verb!x2,y2!), devuelva la resta de ambos (debe devolver un par de
 valores).

    \item Utilizando las funciones anteriores, escribir una función que dados dos
 puntos en el plano (\verb!x1,y1! y \verb!x2,y2!), devuelva la distancia
 entre ambos.

    \item Escribir una función que reciba un vector al origen (definido por sus
 coordenadas \verb!x,y!), y devuelva el vector normalizado correspondiente (debe
 devolver un par de valores \verb!x',y'!).

    \item Utilizando las funciones anteriores (b y d), escribir una función que
 dados dos puntos en el plano (\verb!x1,y1! y \verb!x2,y2!), devuelva el
 vector dirección unitario correspondiente a la recta que los une.

    \item Escribir una función que reciba un punto (\verb!x,y!), una dirección
 unitaria de una recta (\verb!dx,dy!) y un punto perteneciente a esa recta
 (\verb!cx,cy!) y devuelva la proyección del punto sobre la recta. \\
 {\bf Diseño del algoritmo}:
 \begin{enumerate}
     \setlength{\itemsep}{0pt}
     \setlength{\parsep}{0pt}
     \item Al punto a proyectar (\verb!x,y!) restarle el punto de la recta
 (\verb!cx,cy!)
     \item Obtener la matriz de proyección $P$, dada por:  \\
 $p_{11} = d_x^2$,  $p_{12} = p_{21} = d_x*d_y$, $p_{22} = d_y^2$.
     \item Multiplicar la matriz $P$ por el punto obtenido en el paso 1: \\
 $r_x = p_{11} * x + p_12 * y$, $r_y = p_{21} * x + p_{22} * y$.
     \item Al resultado obtenido sumar el punto restado en el paso 1, y
 devolverlo.
 \end{enumerate}

    \item Escribir una función que calcule el área de un triángulo a partir de
 su base y su altura.

    \item Utilizando las funciones anteriores escribir una función que reciba
 tres puntos en el plano (\verb!x1,y1!, \verb!x2,y2! y \verb!x3,y3!) y
 devuelva el área del triángulo correspondiente.
\end{partes}
\end{ejercicio}
